\documentclass{article}
\usepackage{graphicx} % Required for inserting images

\title{MLB Pitcher Injuries and Contributing Factors}
\author{Ajay Natarajan - Statistics Major, University of Connecticut}
\date{October 30, 2023}

\begin{document}

\maketitle

\begin{abstract}
Analytics have become prominent in usage in professional sports within the last two decades. One of the sports that has seen the most advancement within the field is baseball. However, there are various areas within the game of baseball that have had limited research and which are contributing to major inefficiencies at the professional level. A prominent one of these is injuries, particularly to pitchers. This paper will attempt to use available historical injury data as well as existing player and performance data to find significant factors contributing to pitcher injury, and lead insights on how these can be improved.
\end{abstract}

\section{Introduction}
Statistical analysis in sports has been a burgeoning field over the past few decades. In baseball in particular, "analytics" have become especially prominent, with several professional organizations using advanced statistical methods to achieve success and gain a competitive edge. In baseball, one area where it is commonly used is on the side of pitching, enabling pitchers to optimize spin rate, grip, movement, and deception. However, an area with relatively limited study is pitcher injuries. Pitchers are at a relatively high risk of injury, and it has become a growing issue through recent years. This leads to inefficiencies in contract utilization and overall performance for teams, and to lack of career growth for players. 

My research will attempt to categorize injuries into certain boxes:
\begin{enumerate}
\item Analysis of the injury by body part (such as an elbow, shoulder, or oblique injury). 
\item Analysis of the type of injury (such as ulnar collateral ligament tears, a muscle strain, a bone bruse/break, or another unorthodox injury, such as Thoracic Outlet Syndrome.
\item An analysis of how much time each injury caused the pitcher to miss. As a result, my research inquiries will consist of the following:
\end{enumerate}
\begin{itemize}
  \item Do certain pitcher/performance characteristics cause increased rates of injury overall?
  \item Do certain pitcher/performance characteristics cause increased rates of specific types of injury or to a specific body part?
  \item Can certain pitcher/performance characteristics be used to predict likelihood or type of player injury, and how well would it do so?
\end{itemize}

\end{document}
